\section{Zusammenfassung und Fazit}
Im Laufe dieser Arbeit wurde sich in Maschinelle Lernen eingearbeitet und ein Maschinelles Lernen-Modell für die Erkennung einer Ziffer  einer Sieben-Segment-Anzeige erstellt. Diese Sieben-Segment-Anzeige mit  acht weiteren  Ziffern wird von der dem Highspeed-Kamera-Teststand der Firma benutzt,  um die Qualitätssicherung von Kameras und Software durchzuführen.  In der Arbeit wurde sich an die Vorgehensweise von Maschinellen Lernen-Projekt wie in s. \ref{Auflistung Projekt Ablauf} gehalten. \\
Es wurde sich erst mit der Problemstellung auseinander gesetzt. Die Anzeige sollte bei unterschiedlichen Einstellungen der Aufnahmeparameter s. \ref{Parameter} automatisch und verlässlich erkannt werden. Dieses ist gut gelungen auch, wenn nicht alle möglichen Einstellungen bedacht wurden Kapitel \ref{Kapitel: Diskussion}.  Trotzdem kann an der Arbeit gut erkannt werden, wie zuverlässig sich ein Maschinelles Lernen-Modell für die Erkennung von einer Sieben-Segment-Anzeige trainieren lässt.\\
Damit das Modell überhaupt trainiert werden konnte, mussten als erstes Daten aufgenommen werden. Die Aufnahme der Daten geschah mit DaVis und die Daten wurden dann in Python weiterverarbeitet und mit einem Label versehen Kapitel \ref{Kapitel: Daten}. Dabei wurden die Aufnahmen skaliert und in ein numpy-File zusammengefasst,  um die Datengröße zu verringern und schneller lesbar zu machen s. \ref{Lesegeschwindigkeiten}.\\
Zum Erstellen des Maschinellen Lernen-Projektes wurde sich für die Programmiersprache Python und ihre Bibliothek Keras sowie TensorFlow entschieden s.  Kapitel: \ref{Kapitel: Technik}. Die Motivation,  Python zu benutzten,  kam daher, dass Python eine schon bekannte Programmiersprache ist und mit am häufigsten für Maschinelle Lernen-Projekte verwendet wird.  Die Häufigkeit hat den Vorteil das viele Bücher und Beispiele in Python geschrieben sind. Gerade für jemanden,  der sich in das Maschinelle Lernen einarbeitet,  bietet Python und Keras einen schnellen, gut unterstützten und übersichtlichen Einstieg an.  Außerdem lässt sich TensorFlow leicht in 'Google Colab' verwenden.\\
Danach wurde das Modell mit Keras erstellt.  Bei dem Aufstellen des ML-Modelles wurde auf die Wahl der Aktivierungsfunktion geachtet. Zur Verbesserung des Modelles wurden die Lernrate und die Batch-Größe angepasst.  Es stellte sich heraus,  dass mit einer Vergrößerung der Batch-Größe,  also der Unterteilung der Trainingsdaten in mehr Teilstücke, ein sehr gutes Ergebnis erzielt werden konnte. 
Die Funktion von Keras zum Evaluieren von dem Modell hatte eine Genauigkeit von 100\% beim Erkennen des Validierungssets. \\
Das sehr gute Ergebnis wird vor allem an der Ähnlichkeit der Aufnahmen gelegen haben (s. Kapitel \ref{Kapitel: Diskussion}).  Bei der Aufnahmen der Daten wurden wahrscheinlich nicht genug unterschiedliche Einstellung vorgenommen.  Außerdem gibt es bei einer Ziffer einer Sieben-Segment-Anzeige deutlich weniger Unterschiede wie z.B. bei dem MNIST-Datensatz s.\ref{MNIST}. Dadurch ist Lernaufgabe für den Algorithmus unproblematisch.  Damit das ML-Modell in der Praxis bei LaVision verwendet werden könnte,  müsste noch mit Aufnahmen von unterschiedlichen Kameras und Objektiven trainiert werden.\\
Die Maschinelle Lernen-Methode wurde noch mit dem konventionellen Ansatzes der Bachelor-Praxis-Projektes verglichen,  wobei dort der Standpunkt des Programmes zum Ende der Projektphase betrachtet wurde (s.  Kapitel \ref{Kapitel: Diskussion}).  Das Ergebnis des ML-Modelles war in diesem Zustand besser, aber es wurde auch nur eine Ziffer erkannt und der konventionelle Ansatz hatte auch noch einen Logikfehler. Nach einer Verbesserung funktioniert dieser nun sehr gut.\\
Zum Benutzen eines Maschinellen Lernen-Modelles mit der DaVis Software müsste noch eine Einbindung implementiert werden (Kapitel:  \ref{Kapitel: Ausblick}). Dazu gäbe verschieden Möglichkeit: von der simplen Lösung die Aufnahmen in DaVis zu machen und in Python weiterzuverarbeiten,  oder das Modell in Python zu erstellen und in DaVis ein Option zum Laden des ML-Modelles zu implementieren, oder in DaVis selbst eine Möglichkeit zum Erstellen von Maschinellem Lernen-Modellen zu programmieren. \\
Insgesamt war das Projekt ein guter Einstieg in das Maschinelle Lernen, so konnten alle Aspekte eines ML-Projektes s.\ref{Auflistung Projekt Ablauf} durchgearbeitet werden und an ihnen Erfahrung gesammelt werden.  Die Daten wurden selbstständig erzeugt und auch die Schwierigkeiten, die dadurch entstehen können wurden festgestellt. Außerdem war die Komplexität der Aufgabenstellung im richtigen Maße um erfolgreich ein erstes eigenes Maschinelles Lernen-Modell zu erstellen und erfolgreich zu verbessern.  Der Umfang des Projektes hat auch gut gezeigt das Maschinelles Lernen für diese Art von Aufgabenstellung hervorragend geeignet ist. \\
Schlussendlich zeigt das Projekt, dass die Umsetzung von Maschinellen Lernen mit DaVis eine sinnvolle Überlegung für die Zukunft ist.




