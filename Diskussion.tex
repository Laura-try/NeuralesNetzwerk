\section{Diskussion} \label{Kapitel: Diskussion}
In diesem Kapitel werden die Ergebnisse und Probleme des Maschinellen Lernens konkret auf diese Arbeit bezogen aber auch generell diskutiert.  Außerdem wird ein Vergleich zwischen der Methode des Praxisprojektes (konventioneller Ansatz) und  der Bachelorarbeit (ML-Ansatz) gezogen.
\subsection{Ergebnis und Problem des ML}
In der Arbeit ist aufgefallen, dass das Modell zu schnell zu gut funktioniert hat. Es wurde eine Erkennungsquote von 100 \% erreicht.  Es war vorher mit einer Quote um die 90 \% gerechnet worden. Dieses wäre vergleichbar mit den ersten Ergebnissen des MNIST-Datensatzes gewesen.\\
Das zu gute Ergebnis hat zu einer langen Fehlersuche geführt. Es wurde überprüft, ob aus Versehen das Label mit in den Eingabewerten enthalten war und somit zum trainieren mitgeben wurde.  Dazu wurden stichprobenartig 150 Aufnahmen im Originalzustand nochmals per Hand angeschaut. Dabei wurde kontrolliert, ob nicht die Zielvariable in den Metadaten gespeichert wurde. (In der DaVis Software gibt es die Möglichkeit neben den Bilddaten auch Daten zum Bild, wie Kameratyp, Aufnahmezeit usw. zu speichern). Eine weite Überprüfung war die Ausgabe von Aufnahmen mit ihrem Label und der Vorhersage (s.  Kpt. \ref{Überprüfung}). Auch hierbei wurden keine Auffälligkeiten entdeckt. Es werden die richtigen Zahlen prognostiziert,  die Zielvariable ist auch korrekt und die Ziffern sind nicht in einer Reihenfolge.\\
Als nächstes wurden zehn zufällige Aufnahmen ausgewählt um zu schauen, wie unterschiedlich die Aufnahmen tatsächlich sind. Das Problem könnte sein, dass der Unterschied zwischen den einzelnen Aufnahmen zu gering ist.  Zum Beispiel sind alle Ziffern ungefähr gleich groß. Bei den Aufnahmen wurden keine unterschiedlichen Kameras oder Objektive benutzt. \\
Das Problem hierbei ist, dass es einfach ist für den ML-Algorithmus zu lernen und nicht genug generalisiert,  um in andern Konfigurationen (Kameras und Objektiven) zu arbeiten.  \\
Die Schwierigkeit, nicht ausreichend unterschiedliche Daten für einen ML-Algorithmus zu haben,  ist auf der einen Seite durch die Unerfahrenheit beim Erstellen von Maschinellen Lernen-Projekten entstanden. Hinzu kommt die voreingenommene Art Daten zu erheben. Es wurde zwar schon vorher über die benötigen Daten und die benötigte Unterschiedlichkeit der Aufnahmen nachgedacht und verschiedene Einstellungen verwendet, jedoch wurden auch wichtige Einstellungen vergessen.\\
Für eine Maschinelles Lernen-Projekt ist ein großer Satz von Daten, die schon vor dem Projekt erhoben wurden, nützlich,  um den Einfluss der Voreingenommenheit zu verringern.  Ein weiterer wichtiger Punkt, wenn die Daten schon vorhanden sind, ist der Zeitfaktor.  Dieser kann deutlich reduziert werden, wenn die Daten schon vorhanden sind.  Ein Nachteil von schon vorhanden Daten kann sein,  dass sie nicht mit einem Label versehen sind.  Das Versehen mit Zielvariablen ist eine aufwendige und zeitintensive Arbeit, wenn die Daten nicht in einer Reihenfolge oder schon mit irgendeinem Merkmal (Titel) markiert sind. In dieser Arbeit wurde die Reihenfolge der Aufnahmen genutzt,  um möglichst einfach ein Label zu produzieren. Dieses geht natürlich nur,  wenn die Reihenfolge bekannt ist.  Eine Alternative wäre mit unüberwachtem Maschinellen Lernen zu arbeiten,  um die Daten nicht mit einem Label versehen zu müssen.\\
\subsection{Vergleich der Methoden}
Die konventionelle Methode mit dem Durchlaufen der Segmente, wobei dann die Standorte und Helligkeit der LEDs im An-Zustand gespeichert werden,  hat in dem Stand am Ende des Bachelor-Projektes eine Fehlerquote von 8,3 \% \cite{Becker2021}.  Diese Methode wurde noch weiter ausgearbeitet, um mit mehreren Kameras gleichzeitig zu arbeiten.  Dabei wurde festgestellt, dass es in dem Algorithmus einen Logikfehler gab.  Mittlerweile funktioniert auch diese Methode reibungslos.  In dieser Bachelor-Arbeit wird der Zustand nach Ende des Bachelor-Praxis-Projektes verglichen, weil es sich beides um den ungefähren gleichen zeitlichen Aufwand hielt.\\
Der Vorteil der Methode ist, dass auf neue Kameras nicht extra eingegangen werden muss.  Dadurch dass bei jeder Testreihe erst die Kalibrierung durchläuft, kann jedes Mal auf veränderte Umstände reagiert werden. \\
Zu den Nachteilen gehört, dass diese Art der automatischen Kalibrierung Schwierigkeiten mit stark rauschenden Aufnahmen hat. Der Grenzwert der Methode wird durch eine Formel festgelegt und es kann bei Aufnahmen, wo die Segmentes im leuchtenden Zustand nur einige Counts höher liegen, Probleme bei der Erkennung geben. \\
Ein weiterer Nachteil ist, dass die Kalibrierung jedes Mal ausgeführt wird.  Dieses kostet Zeit. Da die beanspruchte Zeit aber unter einer Minute für die gesamte Kalibrierung liegt und der Test jetzt komplett automatisch läuft,  ist dieses kein Faktor mehr.\\
Die Maschinelle Lernen-Methode hat bei dem Daten zwar eine 100\% Genaugikeit bei der Evaluierung,  aber es wurde nur mit einer Kamera getestet. 
 Die Erkennung funktioniert im Moment auch nur für eine Ziffer. Für eine echte Vergleichbarkeit müsste ein Algorithmus zur Bestimmung der Anzeige alle Ziffern betrachten werden.  Die Daten für das Maschinelle Lernen und für die Fehlerquote für der konventionelle Methode wurden mit derselben Kamera und denselben Objektive aufgenommen und sind insofern vergleichbar.\\
Ein Nachteil des Maschinellen Lernens ist die Menge und Varianz der benötigten Lerndaten.  So müssten für die unterschiedlichen Kameratypen und Objektive Aufnahmen gemacht werden,  die zusätzlich noch jeweils in den genannten Parametern (\ref{Parameter}) variieren.  Diese Arbeit würde wahrscheinlich weitere vier Wochen Zeit in Anspruch nehmen. Doch damit wäre immer noch nicht gesichert, ob das Ganze mit der nächsten neuen Kamera funktioniert.  Dies ist aber kein großes Problem, da bevor eine neue Kamera bei LaVision benutzt wird,  sie grundsätzlich sehr umfangreich getestet wird und in DaVis implementiert werden muss.  Es lässt sich außerdem annehmen,  dass das grundsätzliche Modell nicht neu aufgestellt werden muss, sondern nur erneut trainiert werden muss.\\
Ein weitere Nachteil ist die Verknüpfung der Aufnahme aus DaVis (C++) mit dem ML-Modell (Python). Außerdem müssten die Aufnahmen der einzelnen Ziffern alle in das richtige Maß für den Eingabevektor (in dem Model dieser Arbeit 9.296 Einträge) gebracht werden. Dieses ist zwar aufwendig, müsste aber  nur einmal programmiert werden.\\
Der Vorteil des Maschinellen Lernens ist die Genauigkeit, die die Erkennung hat. So konnte in dieser Arbeit eine 100 \% Quote nach der Evaluationsfunktion erreicht werden,  ohne dass es irgendwelche Vorerfahrungen mit dem Maschinellen Lernen gab.  Die Mustererkennung beim Maschinellen Lernern ist deutlich effizienter als konventionelle Ansätze, insbesondere auch, weil manche Muster extrem schwer zu beschreiben sind.\\
In dem behandeltem Beispiel für die automatische Kalibrierung und Erkennung einer Sieben-Segment-Anzeige war die konventionelle Methode einfacher und schneller. 
Die Fehlerquote ist aber höher und die gesamte Software und Infrastruktur ist für diesen Ansatz ausgelegt. So musste nur eine neue Kalibrierung entwickelt werden, den Algorithmus für die Erkennung gab es schon. Außerdem ist das Verarbeiten von der Aufnahmen bereits implementiert. \\
Es ist schwierig, bei so unterschiedlichen Eingangsvoraussetzung eine objektive Schlussfolgerung zu ziehen.  Doch subjektiv gesehen, wenn es noch kein Lesealgorithmus für die Ziffern gegeben hätte, und eine Verknüpfung von DaVis zu einer Maschinen Lernen-Basis bestanden hätte, gehe ich davon aus, dass die Maschinellen Lernen-Methode effizienter gewesen wäre.  Die Fehlerquote wäre geringer und der Mehraufwand des Datenaufnehmens ist eine Arbeit,  die zwischendurch gemacht werden kann.

